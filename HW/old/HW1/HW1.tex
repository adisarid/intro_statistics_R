% Options for packages loaded elsewhere
\PassOptionsToPackage{unicode}{hyperref}
\PassOptionsToPackage{hyphens}{url}
%
\documentclass[
]{article}
\usepackage{lmodern}
\usepackage{amssymb,amsmath}
\usepackage{ifxetex,ifluatex}
\ifnum 0\ifxetex 1\fi\ifluatex 1\fi=0 % if pdftex
  \usepackage[T1]{fontenc}
  \usepackage[utf8]{inputenc}
  \usepackage{textcomp} % provide euro and other symbols
\else % if luatex or xetex
  \usepackage{unicode-math}
  \defaultfontfeatures{Scale=MatchLowercase}
  \defaultfontfeatures[\rmfamily]{Ligatures=TeX,Scale=1}
\fi
% Use upquote if available, for straight quotes in verbatim environments
\IfFileExists{upquote.sty}{\usepackage{upquote}}{}
\IfFileExists{microtype.sty}{% use microtype if available
  \usepackage[]{microtype}
  \UseMicrotypeSet[protrusion]{basicmath} % disable protrusion for tt fonts
}{}
\makeatletter
\@ifundefined{KOMAClassName}{% if non-KOMA class
  \IfFileExists{parskip.sty}{%
    \usepackage{parskip}
  }{% else
    \setlength{\parindent}{0pt}
    \setlength{\parskip}{6pt plus 2pt minus 1pt}}
}{% if KOMA class
  \KOMAoptions{parskip=half}}
\makeatother
\usepackage{xcolor}
\IfFileExists{xurl.sty}{\usepackage{xurl}}{} % add URL line breaks if available
\IfFileExists{bookmark.sty}{\usepackage{bookmark}}{\usepackage{hyperref}}
\hypersetup{
  pdftitle={Introduction to Statistics and Data Analysis with R - Homework \#1},
  pdfauthor={Adi Sarid and Afek Adler},
  hidelinks,
  pdfcreator={LaTeX via pandoc}}
\urlstyle{same} % disable monospaced font for URLs
\usepackage[margin=1in]{geometry}
\usepackage{graphicx,grffile}
\makeatletter
\def\maxwidth{\ifdim\Gin@nat@width>\linewidth\linewidth\else\Gin@nat@width\fi}
\def\maxheight{\ifdim\Gin@nat@height>\textheight\textheight\else\Gin@nat@height\fi}
\makeatother
% Scale images if necessary, so that they will not overflow the page
% margins by default, and it is still possible to overwrite the defaults
% using explicit options in \includegraphics[width, height, ...]{}
\setkeys{Gin}{width=\maxwidth,height=\maxheight,keepaspectratio}
% Set default figure placement to htbp
\makeatletter
\def\fps@figure{htbp}
\makeatother
\setlength{\emergencystretch}{3em} % prevent overfull lines
\providecommand{\tightlist}{%
  \setlength{\itemsep}{0pt}\setlength{\parskip}{0pt}}
\setcounter{secnumdepth}{-\maxdimen} % remove section numbering

\title{Introduction to Statistics and Data Analysis with R - Homework \#1}
\author{Adi Sarid and Afek Adler}
\date{2020-03-18}

\begin{document}
\maketitle

This homework sheet is due on the 2020-04-01 on 24:00. You may submit
your answers in pairs. Submission will be performed electronically via
the moodle.

We urge you to start solving this sheet as soon as possible and, if you
have any questions, come to visit us in reception hours next week.

The exercise is divided into two parts: Technical (programming in
\texttt{R}) and theoretical.

Submit the following questions:

\begin{itemize}
\tightlist
\item
  Q1 - only 3.6.1.6 (specified later).
\item
  Q2
\item
  Q4
\item
  Q5
\end{itemize}

\hypertarget{technical-programming-in-r}{%
\section{\texorpdfstring{Technical (programming in
\texttt{R})}{Technical (programming in R)}}\label{technical-programming-in-r}}

\hypertarget{question-1}{%
\subsection{Question 1:}\label{question-1}}

Please read the following chapters in \href{https://r4ds.had.co.nz}{R4DS
- https://r4ds.had.co.nz}:

\begin{enumerate}
\def\labelenumi{\arabic{enumi}.}
\tightlist
\item
  \href{https://r4ds.had.co.nz/introduction.html}{Introduction}
\item
  \href{https://r4ds.had.co.nz/explore-intro.html}{Explore -
  Introduction}
\item
  \href{https://r4ds.had.co.nz/data-visualisation.html}{Explore - Data
  visualizations}
\item
  \href{https://r4ds.had.co.nz/workflow-basics.html}{Workflow}
\end{enumerate}

Solve exercise 3.6.1,3.81, 4.4. \textbf{submit the code for question 6
(in 3.6.1)} (``\emph{Recreate the R code necessary to generate the
following graphs}'').

\hypertarget{question-2}{%
\subsection{Question 2:}\label{question-2}}

In this question, you will get acquainted (or reminded of) the following
distributions:

\begin{itemize}
\tightlist
\item
  Normal distribution \(N(\mu, \sigma)\)
\item
  Student's t \(t_{\operatorname{df}}\)
\item
  Chi-square \(\chi^2\)
\end{itemize}

Complete the blanks (\texttt{\_\_\_}) in the following code, to generate
\(n=100\) random values from each of these distributions with:

\begin{itemize}
\tightlist
\item
  Normal with \(\mu=3, \sigma = 1.5\)
\item
  Student's-t with \(\operatorname{df}=10\)
\item
  Chi-square with \(\operatorname{df}=12\)
\end{itemize}

Tip: if you type a \texttt{?} followed by the command name in the
console, you will see its documentation. I.e., type \texttt{?rnorm} to
see the help on the random number generator for the normal distribution.

\hypertarget{complete-the-blanks}{%
\subsubsection{Complete the blanks:}\label{complete-the-blanks}}

\begin{verbatim}
set.seed(0)  # we set the seed of the random generator so that your results will be consistent

random_normal <- rnorm(n = ___, mean = ___, ___ = 1.5)
random_t <- rt(n = ___, df = ___)
random_chi <- rchisq(n = ___, df = ___)
\end{verbatim}

\hypertarget{plot-by-completing-the-blanks}{%
\subsubsection{Plot by completing the
blanks:}\label{plot-by-completing-the-blanks}}

Plot each of these samples using \texttt{ggplot2}. Think, what
\texttt{geom} would you use to plot the distribution of the sample?

\begin{verbatim}

# if you don't have the tidyverse package first install by running
# install.packages("tidyverse")

library(tidyverse) 

all_random_data <- tibble(random_normal, random_t, random_chi)

ggplot(all_random_data, aes(random_normal)) + 
  geom____()

ggplot(all_random_data, aes(random_t)) + 
  ___

ggplot(all_random_data, aes(random_chi)) + 
  ___
\end{verbatim}

\hypertarget{answer-these}{%
\subsubsection{Answer these:}\label{answer-these}}

\begin{enumerate}
\def\labelenumi{\arabic{enumi}.}
\item
  Is the original distribution symmetric? does the plots look symmetric,
  why?
\item
  Generally speaking (not relating to the specific sample you obtained),
  what is the relationship between the mean and median of each of these
  distributions?
\item
  What would happen if we increase \(n\) from 100 to 1000?

  \begin{enumerate}
  \def\labelenumii{\alph{enumii}.}
  \tightlist
  \item
    How would the distribution look like?
  \item
    Why?
  \item
    Modify your code and visualize the updates.
  \end{enumerate}
\end{enumerate}

\hypertarget{theoretical}{%
\section{Theoretical}\label{theoretical}}

\hypertarget{question-3}{%
\subsection{Question 3:}\label{question-3}}

In the smallest branch of the smallest bank, the number of customers in
the queue (waiting customers), is a random variable \(Q\in\{0,1,2\}\).
You cannot have more than 2 customers waiting in the queue, because
the've been downsizing and the branch is really small.

The distribution of \(Q\) is dependent on a parameter \(\theta\).

\[Q = \left\{\begin{array}{ll}0 & \text{w.p. }4\theta^2\\
1 & \text{w.p. }4\theta-8\theta^2\\
2 & \text{w.p. }1-4\theta+4\theta^2\end{array}\right.\]

The bank's headquarters randomly sampled the queue during five
independent times. The results were \(\{0,1,0,0,0\}\) customers in the
queue.

\hypertarget{answer-the-following-questions}{%
\subsubsection{Answer the following
questions:}\label{answer-the-following-questions}}

\begin{enumerate}
\def\labelenumi{\arabic{enumi}.}
\tightlist
\item
  Find an unbiased estimator \(\hat{\Theta}\) for the parameter
  \(\theta\) for a sample of size \(n=5\). What is \(\hat{\Theta}\)
  based on the current sample? (you should get 0.45)
\item
  Find an unbiased estimator for the expected number of customers
  waiting in the queue based on a sample of size \(n=5\). What is the
  estimate of the expected number of customers, based on the current
  sample? (0.2)
\item
  Find an estimator for \(\theta\) in the maximum likelihood estimation
  method. (0.45)
\end{enumerate}

\hypertarget{question-4}{%
\subsection{Question 4:}\label{question-4}}

let \(X\) be a random Bernoulli variable. It's probability density
function can be formulated as follows:

\[
f(x ; p)=\left\{\begin{array}{ll}{p^{x}(1-p)^{1-x}} & {x=0,1} \\ {0} & {\text { otherwise }}\end{array}\right.
\]

\begin{enumerate}
\def\labelenumi{\arabic{enumi}.}
\tightlist
\item
  Show that \(X=1\) with probability \(p\) and that \(X=0\) with
  probability \(1-p\)
\item
  Suppose we get a random sample of size \(n\) from a Bernulli
  distribution. What is the likelihood function \(L(p)\) of the sample?
  (what is the probability that
  \(P\left(X_{1}=x_{1}, X_{2}=x_{2}, \ldots, X_{n}=x_{n}\right)\))
\item
  Apply the \(\log\) transformation on this likelihood function, what do
  you get?
\item
  Find the \(p\) that maximizes \(\log L(p)\)
\end{enumerate}

\hypertarget{question-5}{%
\subsection{Question 5:}\label{question-5}}

For the following probability density function:
\[f(x)=\left\{\begin{array}{ll}{\frac{2}{\theta^{2}}(\theta-x)} & {0<x<\theta} \\ {0} & {\text { else }}\end{array}\right.\]
Find \(\theta\) by the method of moments.

\hypertarget{question-6}{%
\subsection{Question 6:}\label{question-6}}

For the exponential distribution:
\[f(x ; \lambda)=\left\{\begin{array}{ll}{\lambda e^{-\lambda x}} & {x \geq 0} \\ {0} & {x<0}\end{array}\right.\]
Find \(\lambda\) by MLE and by the Method of Moments.

\end{document}
